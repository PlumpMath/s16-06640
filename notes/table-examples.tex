\documentclass{article}
\usepackage{multirow}
\begin{document}

A primitive table using only essential components:

% The letters in the second block indicate the alignment of each column
% c = centered, l = left, r = right
% one letter for each column
\begin{tabular}{ccc}

  % We enter text into a column by, and then indicate to LaTeX where the
  % proper alignment is using '&', the same way we did using equations
  A  &  B  &  C \\

  \hline % Add a horizontal line at the bottom of this column

  1  &  2  &  3  \\ 

\end{tabular}

\bigskip



% First we define the formatting we need.
% Four c's indicate that we want 4 centered columns
% Each column will be separated with a solid | since we've added that as well

A basic table using the multicolumn function:

\begin{tabular}{|c|c|c|c|}

  \hline
    
  % The multicolumn tool allows you to join columns
  % First bracket = number of columns to join
  % Second bracket = column style (c for centered)
  % Third column = column entry
  A   &  \multicolumn{3}{c|}{B-2}  \\
    
  \hline
    
  \_  &  1  &  2  &  3  \\ 
    
  \hline  
   
\end{tabular}

\bigskip


A more advanced table removing the boundaries in the upper left:

\begin{tabular}{|*{4}{c|}}  % Functionally equivalent to the syntax above

  \cline{2-4}
    
  \multicolumn{1}{c|}{} & Item1 & Item2 & Item3 \\ \hline
    
  Group1 & 0.8   & 0.1   & 0.1   \\ \hline
    
  Group2 & 0.1   & 0.8   & 0.1   \\ \hline
    
  Group3 & 0.1   & 0.1   & 0.8   \\ \hline
    
  Group4 & 0.34  & 0.33  & 0.33  \\ \hline
    
\end{tabular}


\bigskip




% These tables can quickly become preposterously large, for what feels like very little return.
% Proceed with caution!

An elite table using both the multicolumn and multirow functions:

\begin{tabular}{cc|c|c|c|c|l}
  
  \cline{3-6}
  
  & & \multicolumn{4}{ c| }{Primes} \\ \cline{3-6}
  
  & & 2 & 3 & 5 & 7 \\ \cline{1-6}
  
  \multicolumn{1}{ |c  }{\multirow{2}{*}{Powers} } &
  
  \multicolumn{1}{ |c| }{504} & 3 & 2 & 0 & 1 &     \\ \cline{2-6}
  
  \multicolumn{1}{ |c  }{}                        &
  
  \multicolumn{1}{ |c| }{540} & 2 & 3 & 1 & 0 &     \\ \cline{1-6}
  
  \multicolumn{1}{ |c  }{\multirow{2}{*}{Powers} } &
  
  \multicolumn{1}{ |c| }{gcd} & 2 & 2 & 0 & 0 & min \\ \cline{2-6}
  
  \multicolumn{1}{ |c  }{}                        &
  
  \multicolumn{1}{ |c| }{lcm} & 3 & 3 & 1 & 1 & max \\ \cline{1-6}
  
\end{tabular}

\bigskip

\begin{equation}
A = B * \hat{Cat}
\end{equation}

\end{document}

